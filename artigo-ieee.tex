
%% bare_conf.tex
%% V1.4
%% 2012/12/27
%% by Michael Shell
%% See:
%% http://www.michaelshell.org/
%% for current contact information.
%%
%% This is a skeleton file demonstrating the use of IEEEtran.cls
%% (requires IEEEtran.cls version 1.8 or later) with an IEEE conference paper.
%%
%% Support sites:
%% http://www.michaelshell.org/tex/ieeetran/
%% http://www.ctan.org/tex-archive/macros/latex/contrib/IEEEtran/
%% and
%% http://www.ieee.org/

%%*************************************************************************
%% Legal Notice:
%% This code is offered as-is without any warranty either expressed or
%% implied; without even the implied warranty of MERCHANTABILITY or
%% FITNESS FOR A PARTICULAR PURPOSE! 
%% User assumes all risk.
%% In no event shall IEEE or any contributor to this code be liable for
%% any damages or losses, including, but not limited to, incidental,
%% consequential, or any other damages, resulting from the use or misuse
%% of any information contained here.
%%
%% All comments are the opinions of their respective authors and are not
%% necessarily endorsed by the IEEE.
%%
%% This work is distributed under the LaTeX Project Public License (LPPL)
%% ( http://www.latex-project.org/ ) version 1.3, and may be freely used,
%% distributed and modified. A copy of the LPPL, version 1.3, is included
%% in the base LaTeX documentation of all distributions of LaTeX released
%% 2003/12/01 or later.
%% Retain all contribution notices and credits.
%% ** Modified files should be clearly indicated as such, including  **
%% ** renaming them and changing author support contact information. **
%%
%% File list of work: IEEEtran.cls, IEEEtran_HOWTO.pdf, bare_adv.tex,
%%                    bare_conf.tex, bare_jrnl.tex, bare_jrnl_compsoc.tex,
%%                    bare_jrnl_transmag.tex
%%*************************************************************************

% *** Authors should verify (and, if needed, correct) their LaTeX system  ***
% *** with the testflow diagnostic prior to trusting their LaTeX platform ***
% *** with production work. IEEE's font choices can trigger bugs that do  ***
% *** not appear when using other class files.                            ***
% The testflow support page is at:
% http://www.michaelshell.org/tex/testflow/



% Note that the a4paper option is mainly intended so that authors in
% countries using A4 can easily print to A4 and see how their papers will
% look in print - the typesetting of the document will not typically be
% affected with changes in paper size (but the bottom and side margins will).
% Use the testflow package mentioned above to verify correct handling of
% both paper sizes by the user's LaTeX system.
%
% Also note that the "draftcls" or "draftclsnofoot", not "draft", option
% should be used if it is desired that the figures are to be displayed in
% draft mode.
%
\documentclass[conference]{IEEEtran}
% Add the compsoc option for Computer Society conferences.
%
% If IEEEtran.cls has not been installed into the LaTeX system files,
% manually specify the path to it like:
% \documentclass[conference]{../sty/IEEEtran}





% Some very useful LaTeX packages include:
% (uncomment the ones you want to load)


% *** MISC UTILITY PACKAGES ***
%
%\usepackage{ifpdf}
% Heiko Oberdiek's ifpdf.sty is very useful if you need conditional
% compilation based on whether the output is pdf or dvi.
% usage:
% \ifpdf
%   % pdf code
% \else
%   % dvi code
% \fi
% The latest version of ifpdf.sty can be obtained from:
% http://www.ctan.org/tex-archive/macros/latex/contrib/oberdiek/
% Also, note that IEEEtran.cls V1.7 and later provides a builtin
% \ifCLASSINFOpdf conditional that works the same way.
% When switching from latex to pdflatex and vice-versa, the compiler may
% have to be run twice to clear warning/error messages.






% *** CITATION PACKAGES ***
%
%\usepackage{cite}
% cite.sty was written by Donald Arseneau
% V1.6 and later of IEEEtran pre-defines the format of the cite.sty package
% \cite{} output to follow that of IEEE. Loading the cite package will
% result in citation numbers being automatically sorted and properly
% "compressed/ranged". e.g., [1], [9], [2], [7], [5], [6] without using
% cite.sty will become [1], [2], [5]--[7], [9] using cite.sty. cite.sty's
% \cite will automatically add leading space, if needed. Use cite.sty's
% noadjust option (cite.sty V3.8 and later) if you want to turn this off
% such as if a citation ever needs to be enclosed in parenthesis.
% cite.sty is already installed on most LaTeX systems. Be sure and use
% version 4.0 (2003-05-27) and later if using hyperref.sty. cite.sty does
% not currently provide for hyperlinked citations.
% The latest version can be obtained at:
% http://www.ctan.org/tex-archive/macros/latex/contrib/cite/
% The documentation is contained in the cite.sty file itself.






% *** GRAPHICS RELATED PACKAGES ***
%
\ifCLASSINFOpdf
  % \usepackage[pdftex]{graphicx}
  % declare the path(s) where your graphic files are
  % \graphicspath{{../pdf/}{../jpeg/}}
  % and their extensions so you won't have to specify these with
  % every instance of \includegraphics
  % \DeclareGraphicsExtensions{.pdf,.jpeg,.png}
\else
  % or other class option (dvipsone, dvipdf, if not using dvips). graphicx
  % will default to the driver specified in the system graphics.cfg if no
  % driver is specified.
  % \usepackage[dvips]{graphicx}
  % declare the path(s) where your graphic files are
  % \graphicspath{{../eps/}}
  % and their extensions so you won't have to specify these with
  % every instance of \includegraphics
  % \DeclareGraphicsExtensions{.eps}
\fi
% graphicx was written by David Carlisle and Sebastian Rahtz. It is
% required if you want graphics, photos, etc. graphicx.sty is already
% installed on most LaTeX systems. The latest version and documentation
% can be obtained at: 
% http://www.ctan.org/tex-archive/macros/latex/required/graphics/
% Another good source of documentation is "Using Imported Graphics in
% LaTeX2e" by Keith Reckdahl which can be found at:
% http://www.ctan.org/tex-archive/info/epslatex/
%
% latex, and pdflatex in dvi mode, support graphics in encapsulated
% postscript (.eps) format. pdflatex in pdf mode supports graphics
% in .pdf, .jpeg, .png and .mps (metapost) formats. Users should ensure
% that all non-photo figures use a vector format (.eps, .pdf, .mps) and
% not a bitmapped formats (.jpeg, .png). IEEE frowns on bitmapped formats
% which can result in "jaggedy"/blurry rendering of lines and letters as
% well as large increases in file sizes.
%
% You can find documentation about the pdfTeX application at:
% http://www.tug.org/applications/pdftex





% *** MATH PACKAGES ***
%
%\usepackage[cmex10]{amsmath}
% A popular package from the American Mathematical Society that provides
% many useful and powerful commands for dealing with mathematics. If using
% it, be sure to load this package with the cmex10 option to ensure that
% only type 1 fonts will utilized at all point sizes. Without this option,
% it is possible that some math symbols, particularly those within
% footnotes, will be rendered in bitmap form which will result in a
% document that can not be IEEE Xplore compliant!
%
% Also, note that the amsmath package sets \interdisplaylinepenalty to 10000
% thus preventing page breaks from occurring within multiline equations. Use:
%\interdisplaylinepenalty=2500
% after loading amsmath to restore such page breaks as IEEEtran.cls normally
% does. amsmath.sty is already installed on most LaTeX systems. The latest
% version and documentation can be obtained at:
% http://www.ctan.org/tex-archive/macros/latex/required/amslatex/math/





% *** SPECIALIZED LIST PACKAGES ***
%
%\usepackage{algorithmic}
% algorithmic.sty was written by Peter Williams and Rogerio Brito.
% This package provides an algorithmic environment fo describing algorithms.
% You can use the algorithmic environment in-text or within a figure
% environment to provide for a floating algorithm. Do NOT use the algorithm
% floating environment provided by algorithm.sty (by the same authors) or
% algorithm2e.sty (by Christophe Fiorio) as IEEE does not use dedicated
% algorithm float types and packages that provide these will not provide
% correct IEEE style captions. The latest version and documentation of
% algorithmic.sty can be obtained at:
% http://www.ctan.org/tex-archive/macros/latex/contrib/algorithms/
% There is also a support site at:
% http://algorithms.berlios.de/index.html
% Also of interest may be the (relatively newer and more customizable)
% algorithmicx.sty package by Szasz Janos:
% http://www.ctan.org/tex-archive/macros/latex/contrib/algorithmicx/




% *** ALIGNMENT PACKAGES ***
%
%\usepackage{array}
% Frank Mittelbach's and David Carlisle's array.sty patches and improves
% the standard LaTeX2e array and tabular environments to provide better
% appearance and additional user controls. As the default LaTeX2e table
% generation code is lacking to the point of almost being broken with
% respect to the quality of the end results, all users are strongly
% advised to use an enhanced (at the very least that provided by array.sty)
% set of table tools. array.sty is already installed on most systems. The
% latest version and documentation can be obtained at:
% http://www.ctan.org/tex-archive/macros/latex/required/tools/


% IEEEtran contains the IEEEeqnarray family of commands that can be used to
% generate multiline equations as well as matrices, tables, etc., of high
% quality.




% *** SUBFIGURE PACKAGES ***
%\ifCLASSOPTIONcompsoc
%  \usepackage[caption=false,font=normalsize,labelfont=sf,textfont=sf]{subfig}
%\else
%  \usepackage[caption=false,font=footnotesize]{subfig}
%\fi
% subfig.sty, written by Steven Douglas Cochran, is the modern replacement
% for subfigure.sty, the latter of which is no longer maintained and is
% incompatible with some LaTeX packages including fixltx2e. However,
% subfig.sty requires and automatically loads Axel Sommerfeldt's caption.sty
% which will override IEEEtran.cls' handling of captions and this will result
% in non-IEEE style figure/table captions. To prevent this problem, be sure
% and invoke subfig.sty's "caption=false" package option (available since
% subfig.sty version 1.3, 2005/06/28) as this is will preserve IEEEtran.cls
% handling of captions.
% Note that the Computer Society format requires a larger sans serif font
% than the serif footnote size font used in traditional IEEE formatting
% and thus the need to invoke different subfig.sty package options depending
% on whether compsoc mode has been enabled.
%
% The latest version and documentation of subfig.sty can be obtained at:
% http://www.ctan.org/tex-archive/macros/latex/contrib/subfig/




% *** FLOAT PACKAGES ***
%
%\usepackage{fixltx2e}
% fixltx2e, the successor to the earlier fix2col.sty, was written by
% Frank Mittelbach and David Carlisle. This package corrects a few problems
% in the LaTeX2e kernel, the most notable of which is that in current
% LaTeX2e releases, the ordering of single and double column floats is not
% guaranteed to be preserved. Thus, an unpatched LaTeX2e can allow a
% single column figure to be placed prior to an earlier double column
% figure. The latest version and documentation can be found at:
% http://www.ctan.org/tex-archive/macros/latex/base/


%\usepackage{stfloats}
% stfloats.sty was written by Sigitas Tolusis. This package gives LaTeX2e
% the ability to do double column floats at the bottom of the page as well
% as the top. (e.g., "\begin{figure*}[!b]" is not normally possible in
% LaTeX2e). It also provides a command:
%\fnbelowfloat
% to enable the placement of footnotes below bottom floats (the standard
% LaTeX2e kernel puts them above bottom floats). This is an invasive package
% which rewrites many portions of the LaTeX2e float routines. It may not work
% with other packages that modify the LaTeX2e float routines. The latest
% version and documentation can be obtained at:
% http://www.ctan.org/tex-archive/macros/latex/contrib/sttools/
% Do not use the stfloats baselinefloat ability as IEEE does not allow
% \baselineskip to stretch. Authors submitting work to the IEEE should note
% that IEEE rarely uses double column equations and that authors should try
% to avoid such use. Do not be tempted to use the cuted.sty or midfloat.sty
% packages (also by Sigitas Tolusis) as IEEE does not format its papers in
% such ways.
% Do not attempt to use stfloats with fixltx2e as they are incompatible.
% Instead, use Morten Hogholm'a dblfloatfix which combines the features
% of both fixltx2e and stfloats:
%
% \usepackage{dblfloatfix}
% The latest version can be found at:
% http://www.ctan.org/tex-archive/macros/latex/contrib/dblfloatfix/




% *** PDF, URL AND HYPERLINK PACKAGES ***
%
%\usepackage{url}
% url.sty was written by Donald Arseneau. It provides better support for
% handling and breaking URLs. url.sty is already installed on most LaTeX
% systems. The latest version and documentation can be obtained at:
% http://www.ctan.org/tex-archive/macros/latex/contrib/url/
% Basically, \url{my_url_here}.

%*** para suportar acentuação ***
\usepackage[utf8]{inputenc}

%*** para suportar tabelas com colunas mergeadas ***
\usepackage{multirow}

%*** Para inclusão de imagens e permitir rotacionar texto ***
\usepackage{graphicx}			% Inclusão de gráficos
\graphicspath{ {./} }			% localizando as imagens

%*** Para ajustar a largura das colunas e para multilinhas nas células ***
\usepackage{array}
\newcolumntype{L}{>{\centering\arraybackslash}m{1cm}}
\newcolumntype{M}{>{\centering\arraybackslash}m{3cm}}

% *** Do not adjust lengths that control margins, column widths, etc. ***
% *** Do not use packages that alter fonts (such as pslatex).         ***
% There should be no need to do such things with IEEEtran.cls V1.6 and later.
% (Unless specifically asked to do so by the journal or conference you plan
% to submit to, of course. )


% correct bad hyphenation here
\hyphenation{op-tical net-works semi-conduc-tor}


\begin{document}
%
% paper title
% can use linebreaks \\ within to get better formatting as desired
% Do not put math or special symbols in the title.
\title{Coletando dados de memória de uma máquina em nuvem para análise forense}


% author names and affiliations
% use a multiple column layout for up to three different
% affiliations
\author{\IEEEauthorblockN{Hamilton Fonte II}
\IEEEauthorblockA{Universidade de São Paulo (USP)\\
Escola Politécnica - Engenharia de Computação\\
Programa de Pós Graduação em Engenharia Elétrica\\
São Paulo, SP, Brasil\\
Email: hamiltonii@gmail.com}
\and
\IEEEauthorblockN{Marcus Simplício Jr.}
\IEEEauthorblockA{Orientador \\
Universidade de São Paulo (USP)\\
Escola Politécnica - Engenharia de Computação\\
Programa de Pós Graduação em Engenharia Elétrica\\
São Paulo, SP, Brasil}
}

% conference papers do not typically use \thanks and this command
% is locked out in conference mode. If really needed, such as for
% the acknowledgment of grants, issue a \IEEEoverridecommandlockouts
% after \documentclass

% for over three affiliations, or if they all won't fit within the width
% of the page, use this alternative format:
% 
%\author{\IEEEauthorblockN{Michael Shell\IEEEauthorrefmark{1},
%Homer Simpson\IEEEauthorrefmark{2},
%James Kirk\IEEEauthorrefmark{3}, 
%Montgomery Scott\IEEEauthorrefmark{3} and
%Eldon Tyrell\IEEEauthorrefmark{4}}
%\IEEEauthorblockA{\IEEEauthorrefmark{1}School of Electrical and Computer Engineering\\
%Georgia Institute of Technology,
%Atlanta, Georgia 30332--0250\\ Email: see http://www.michaelshell.org/contact.html}
%\IEEEauthorblockA{\IEEEauthorrefmark{2}Twentieth Century Fox, Springfield, USA\\
%Email: homer@thesimpsons.com}
%\IEEEauthorblockA{\IEEEauthorrefmark{3}Starfleet Academy, San Francisco, California 96678-2391\\
%Telephone: (800) 555--1212, Fax: (888) 555--1212}
%\IEEEauthorblockA{\IEEEauthorrefmark{4}Tyrell Inc., 123 Replicant Street, Los Angeles, California 90210--4321}}




% use for special paper notices
%\IEEEspecialpapernotice{(Invited Paper)}




% make the title area
\maketitle

% As a general rule, do not put math, special symbols or citations
% in the abstract
\begin{abstract}
Lorem ipsum dolor sit amet, consectetuer adipiscing elit. Aenean commodo ligula eget dolor. Aenean massa. Cum sociis natoque penatibus et magnis dis parturient montes, 
nascetur ridiculus mus. Donec quam felis, ultricies nec, pellentesque eu, pretium quis, sem. Nulla consequat massa quis enim. Donec pede justo, fringilla vel, aliquet 
nec, vulputate eget, arcu. In enim justo, rhoncus ut, imperdiet a, venenatis vitae, justo. Nullam dictum felis eu pede mollis pretium.
\end{abstract}

% no keywords




% For peer review papers, you can put extra information on the cover
% page as needed:
% \ifCLASSOPTIONpeerreview
% \begin{center} \bfseries EDICS Category: 3-BBND \end{center}
% \fi
%
% For peerreview papers, this IEEEtran command inserts a page break and
% creates the second title. It will be ignored for other modes.
\IEEEpeerreviewmaketitle



\section{Introdução}
% no \IEEEPARstart
Lorem ipsum dolor sit amet, consectetuer adipiscing elit. Aenean commodo ligula eget dolor. Aenean massa. Cum sociis natoque penatibus et magnis dis parturient montes, 
nascetur ridiculus mus. Donec quam felis, ultricies nec, pellentesque eu, pretium quis, sem. Nulla consequat massa quis enim. Donec pede justo, fringilla vel, aliquet 
nec, vulputate eget, arcu. In enim justo, rhoncus ut, imperdiet a, venenatis vitae, justo. Nullam dictum felis eu pede mollis pretium. Integer tincidunt. Cras dapibus.
Vivamus elementum semper nisi. Aenean vulputate eleifend tellus. Aenean leo ligula, porttitor eu, consequat vitae, eleifend ac, enim. Aliquam lorem ante, dapibus in, 
viverra quis, feugiat a, tellus. Phasellus viverra nulla ut metus varius laoreet. Quisque rutrum. Aenean imperdiet. Etiam ultricies nisi vel augue. Curabitur ullamcorper 
ultricies nisi. Nam eget dui. Etiam rhoncus. Maecenas tempus, tellus eget condimentum rhoncus, sem quam semper libero, sit amet adipiscing sem neque sed ipsum. Nam quam 
nunc, blandit vel, luctus pulvinar, hendrerit id, lorem. Maecenas nec odio et ante tincidunt tempus. Donec vitae sapien ut libero venenatis faucibus. Nullam quis ante. 
Etiam sit amet orci eget eros faucibus tincidunt. Duis leo. Sed fringilla mauris sit amet nibh. Donec sodales sagittis magna. Sed consequat, leo eget bibendum sodales, 
augue velit cursus nunc, 
% You must have at least 2 lines in the paragraph with the drop letter
% (should never be an issue)
I wish you the best of success.

\hfill mds
 
\hfill December 27, 2012

\section{Computação em nuvem}

\subsection{Virtualização}

Lorem ipsum dolor sit amet, consectetuer adipiscing elit. Aenean commodo ligula eget dolor. Aenean massa. Cum sociis natoque penatibus et magnis dis parturient montes, 
nascetur ridiculus mus. Donec quam felis, ultricies nec, pellentesque eu, pretium quis, sem. Nulla consequat massa quis enim. Donec pede justo, fringilla vel, aliquet 
nec, vulputate eget, arcu. In enim justo, rhoncus ut, imperdiet a, venenatis vitae, justo. Nullam dictum felis eu pede mollis pretium. Integer tincidunt. Cras dapibus.
Vivamus elementum semper nisi. Aenean vulputate eleifend tellus. Aenean leo ligula, porttitor eu, consequat vitae, eleifend ac, enim. Aliquam lorem ante, dapibus in, 
viverra quis, feugiat a, tellus. Phasellus viverra nulla ut metus varius laoreet. Quisque rutrum. Aenean imperdiet. Etiam ultricies nisi vel augue. Curabitur ullamcorper 
ultricies nisi. Nam eget dui. Etiam rhoncus. Maecenas tempus, tellus eget condimentum rhoncus, sem quam semper libero, sit amet adipiscing sem neque sed ipsum. Nam quam 
nunc, blandit vel, luctus pulvinar, hendrerit id, lorem. Maecenas nec odio et ante tincidunt tempus. Donec vitae sapien ut libero venenatis faucibus. Nullam quis ante. 
Etiam sit amet orci eget eros faucibus tincidunt. Duis leo. Sed fringilla mauris sit amet nibh. Donec sodales sagittis magna. Sed consequat, leo eget bibendum sodales, 
augue velit cursus nunc, 

I wish you the best of success.

\hfill mds
 
\hfill December 27, 2012

\subsection{Containerização}

Lorem ipsum dolor sit amet, consectetuer adipiscing elit. Aenean commodo ligula eget dolor. Aenean massa. Cum sociis natoque penatibus et magnis dis parturient montes, 
nascetur ridiculus mus. Donec quam felis, ultricies nec, pellentesque eu, pretium quis, sem. Nulla consequat massa quis enim. Donec pede justo, fringilla vel, aliquet 
nec, vulputate eget, arcu. In enim justo, rhoncus ut, imperdiet a, venenatis vitae, justo. Nullam dictum felis eu pede mollis pretium. Integer tincidunt. Cras dapibus.
Vivamus elementum semper nisi. Aenean vulputate eleifend tellus. Aenean leo ligula, porttitor eu, consequat vitae, eleifend ac, enim. Aliquam lorem ante, dapibus in, 
viverra quis, feugiat a, tellus. Phasellus viverra nulla ut metus varius laoreet. Quisque rutrum. Aenean imperdiet. Etiam ultricies nisi vel augue. Curabitur ullamcorper 
ultricies nisi. Nam eget dui. Etiam rhoncus. Maecenas tempus, tellus eget condimentum rhoncus, sem quam semper libero, sit amet adipiscing sem neque sed ipsum. Nam quam 
nunc, blandit vel, luctus pulvinar, hendrerit id, lorem. Maecenas nec odio et ante tincidunt tempus. Donec vitae sapien ut libero venenatis faucibus. Nullam quis ante. 
Etiam sit amet orci eget eros faucibus tincidunt. Duis leo. Sed fringilla mauris sit amet nibh. Donec sodales sagittis magna. Sed consequat, leo eget bibendum sodales, 
augue velit cursus nunc, 

I wish you the best of success.

\hfill mds
 
\hfill December 27, 2012

\section{Forense de memória em Nuvem}

\subsection{Volatilidade das máquinas}

Lorem ipsum dolor sit amet, consectetuer adipiscing elit. Aenean commodo ligula eget dolor. Aenean massa. Cum sociis natoque penatibus et magnis dis parturient montes, 
nascetur ridiculus mus. Donec quam felis, ultricies nec, pellentesque eu, pretium quis, sem. Nulla consequat massa quis enim. Donec pede justo, fringilla vel, aliquet 
nec, vulputate eget, arcu. In enim justo, rhoncus ut, imperdiet a, venenatis vitae, justo. Nullam dictum felis eu pede mollis pretium. Integer tincidunt. Cras dapibus.
Vivamus elementum semper nisi. Aenean vulputate eleifend tellus. Aenean leo ligula, porttitor eu, consequat vitae, eleifend ac, enim. Aliquam lorem ante, dapibus in, 
viverra quis, feugiat a, tellus. Phasellus viverra nulla ut metus varius laoreet. Quisque rutrum. Aenean imperdiet. Etiam ultricies nisi vel augue. Curabitur ullamcorper 
ultricies nisi. Nam eget dui. Etiam rhoncus. Maecenas tempus, tellus eget condimentum rhoncus, sem quam semper libero, sit amet adipiscing sem neque sed ipsum. Nam quam 
nunc, blandit vel, luctus pulvinar, hendrerit id, lorem. Maecenas nec odio et ante tincidunt tempus. Donec vitae sapien ut libero venenatis faucibus. Nullam quis ante. 
Etiam sit amet orci eget eros faucibus tincidunt. Duis leo. Sed fringilla mauris sit amet nibh. Donec sodales sagittis magna. Sed consequat, leo eget bibendum sodales, 
augue velit cursus nunc, 

I wish you the best of success.

\hfill mds
 
\hfill December 27, 2012

\subsection{Coleta dos dados}

Lorem ipsum dolor sit amet, consectetuer adipiscing elit. Aenean commodo ligula eget dolor. Aenean massa. Cum sociis natoque penatibus et magnis dis parturient montes, 
nascetur ridiculus mus. Donec quam felis, ultricies nec, pellentesque eu, pretium quis, sem. Nulla consequat massa quis enim. Donec pede justo, fringilla vel, aliquet 
nec, vulputate eget, arcu. In enim justo, rhoncus ut, imperdiet a, venenatis vitae, justo. Nullam dictum felis eu pede mollis pretium. Integer tincidunt. Cras dapibus.
Vivamus elementum semper nisi. Aenean vulputate eleifend tellus. Aenean leo ligula, porttitor eu, consequat vitae, eleifend ac, enim. Aliquam lorem ante, dapibus in, 
viverra quis, feugiat a, tellus. Phasellus viverra nulla ut metus varius laoreet. Quisque rutrum. Aenean imperdiet. Etiam ultricies nisi vel augue. Curabitur ullamcorper 
ultricies nisi. Nam eget dui. Etiam rhoncus. Maecenas tempus, tellus eget condimentum rhoncus, sem quam semper libero, sit amet adipiscing sem neque sed ipsum. Nam quam 
nunc, blandit vel, luctus pulvinar, hendrerit id, lorem. Maecenas nec odio et ante tincidunt tempus. Donec vitae sapien ut libero venenatis faucibus. Nullam quis ante. 
Etiam sit amet orci eget eros faucibus tincidunt. Duis leo. Sed fringilla mauris sit amet nibh. Donec sodales sagittis magna. Sed consequat, leo eget bibendum sodales, 
augue velit cursus nunc, 

I wish you the best of success.

\hfill mds
 
\hfill December 27, 2012

\subsection{Cadeia de custódia}

Lorem ipsum dolor sit amet, consectetuer adipiscing elit. Aenean commodo ligula eget dolor. Aenean massa. Cum sociis natoque penatibus et magnis dis parturient montes, 
nascetur ridiculus mus. Donec quam felis, ultricies nec, pellentesque eu, pretium quis, sem. Nulla consequat massa quis enim. Donec pede justo, fringilla vel, aliquet 
nec, vulputate eget, arcu. In enim justo, rhoncus ut, imperdiet a, venenatis vitae, justo. Nullam dictum felis eu pede mollis pretium. Integer tincidunt. Cras dapibus.
Vivamus elementum semper nisi. Aenean vulputate eleifend tellus. Aenean leo ligula, porttitor eu, consequat vitae, eleifend ac, enim. Aliquam lorem ante, dapibus in, 
viverra quis, feugiat a, tellus. Phasellus viverra nulla ut metus varius laoreet. Quisque rutrum. Aenean imperdiet. Etiam ultricies nisi vel augue. Curabitur ullamcorper 
ultricies nisi. Nam eget dui. Etiam rhoncus. Maecenas tempus, tellus eget condimentum rhoncus, sem quam semper libero, sit amet adipiscing sem neque sed ipsum. Nam quam 
nunc, blandit vel, luctus pulvinar, hendrerit id, lorem. Maecenas nec odio et ante tincidunt tempus. Donec vitae sapien ut libero venenatis faucibus. Nullam quis ante. 
Etiam sit amet orci eget eros faucibus tincidunt. Duis leo. Sed fringilla mauris sit amet nibh. Donec sodales sagittis magna. Sed consequat, leo eget bibendum sodales, 
augue velit cursus nunc, 

I wish you the best of success.

\hfill mds
 
\hfill December 27, 2012

\section{Trabalhos relacionados}

\begin{itemize}

\item \textbf{Digital forensics framework for a cloud environment \cite{George2012} }: Arcabouço para coleta de dados de máquinas virtuais. Possui duas formas de 
acionamento, a manual e a automática, integrada com algum sistema de detecção de ameaça. Quando acionado, escuta a rede, determina qual máquina é objeto de investigação, 
coleta informações de log e tráfego de rede e associa ao usuário da respectiva máquina. Propõe o armazenamento das evidências em local fora da núvem para escapar de 
problemas de jurisdição e multi-inquilino mas tem inteligência para usar a própria nuvem como armazenamento caso o espaço fora acabe.

A proposta dá a entender que é aplicável apenas a um sistema virtual estático, onde o número e organização das máquinas é constante. De informação volátil coleta apenas 
tráfego de rede, não coleta memória. Com a forma de acionamento descrito ele não consegue descrever, com as evidências, como era o sistema antes do ataque. Apesar de armazenar 
a evidência fora da nuvem, não da detalhes de como garante que a evidência não foi alterada ou destruída no transporte até o local de armazenamento nem como controla o 
acesso a evidência.

Quando comparado a este trabalho, a presente proposta tem por vantagens a utilização de container para associar a evidência a sua origem tornando o processo independente 
de máquina e permitindo que seja repetido mesmo se a máquina de onde se originaram os dados não existir mais. Com a implementação de uma janela de x dias de coleta antes da
detecção do ataque é possível descrever, através de evidência, como era o sistema antes do mesmo. Com isso a solução apresentada consegue evidências em um cenário de
infra-estrutura dinâmica. São tomadas precauções para garantir que os dados não foram alterados ou destruídos no transporte via TLS os para um local fora da nuvem e 
o acesso a mesma é controlado.\\

\item \textbf{Evidence and cloud computing the virtual machine introspection approach \cite{Poisel2013} }: Descreve um método de coleta de informações de máquinas
em nuvem através da técnica de introspecção em máquina virtual, onde se acessa os dados das máquinas virtuais através do hypervisor. Propõe que o processo seja disparado
automaticamente integrado a um sistema de detecção de ameaça mas também suporta acionamento manual.

A técnica descrita cobre apenas o processo de coleta de informações, não explica onde ou como elas serão armazenadas. No que tange as informações de memória, como os 
endereços de memória são os do host, estes precisam ser traduzidos para que a análise forense seja feita. Segundo a comunidade, tal estratégia é imune a técnicas anti-
forenses empregadas por usuários maliciosos pois está localizada fora da máquina virtual. Como a abordagem não tem conhecimento do que está rodando dentro da máquina 
precisa de uma copia bit a bit da evidência. Embora pareça possível, não descreve como lida com o cenário onde uma máquina é despejada do pool e os recursos liberados. 

Quando comparado a este trabalho, a presente proposta tem por vantagens ser um arcabouço para coleta e armazenamento de evidências. Usa-se uma estratégia 
diferente pois coleta-se a memória diretamente de dentro da máquina virtual onde se evita o problema do gap semântico próprio das soluções por introspecção. 
Como não precisa realizar tradução de endereços, a presente proposta consegue realizar uma coleta onde os dados já são úteis para análise e pode direcionar a mesma
pois tem o conhecimento do que está rodando na máquina. De acordo com a comunidade é mais sucetível a técnicas anti-forenses.\\

\item \textbf{Design and implementation of FROST: FoRensic tools for Open STack \cite{Dykstra2013} }: Arcabouço para coleta de dados de máquinas virtuais através da API do
hypervisor. Isola a máquina virtual afetada do pool original para realização da coleta. Precisa ser acionado quando uma ameaça é detectada. É o mais bem acabado arcabouço de
todas as propostas encontradas até agora mas ao detalhar o processo de armazenamento não explica como garante que a evidência não será destruida ou alterada no transporte
até o armazenamento nem como controla o acesso a evidência. Por estar integrado ao Open Stack o arcabouço depende de cooperação do provedor de serviços de nuvem onde ele 
está rodando, isso é considerado problemático pela comunidade pois a prioridade do mesmo é manter o serviço funcionando e não coletar evidencias forenses. Como está na
mesma camada do hypervisor não conhece o que está rodando dentro da máquina. Depende da existência da máquina virtual para realização da coleta.

Quando comparado a este trabalho, a presente proposta tem por vantagens a utilização de container para associar a evidência a sua origem tornando o processo 
independente de máquina e permitindo que seja repetido mesmo se a máquina de onde originaram os dados não existir mais. Coma implementação de uma janela 
de x dias de coleta antes da detecção do ataque é possível descrever, através da evidência, como era o sistema antes do mesmo. Não depende de cooperação do provedor 
do serviço de nuvem. A presente proposta também consegue realizar uma coleta onde os dados já são úteis para análise e pode direcionar a mesma pois tem o conhecimento 
do que está rodando na máquina. \\

\item \textbf{Automated Forensic Data Acquisition in the Cloud \cite{Reichert2015} }: Propõe um modelo que tira instantâneos de máquinas virtuais atrelado a algum mecanismo de
detecção de ameaça baseado no hypervisor. Usa o Google Rapid Response para salvar as informações coletadas fora da núvem de forma a driblar os problemas de multi-jurisdição e 
multi-inquilino. Descreve satisfatóriamente como evita que a evidência seja alterada ou destruída no transporte até o armazenamento e como controla o acesso a evidência.

O modelo proposto só começa a coletar evidência após a detecção da ameaça e toma um instantâneo da máquina toda o que já foi julgado pela comunidade como um processo custoso em
termos de espaco em disco e piora o problema do volume de dados a ser analisado. Pessoalmente acho arriscado depender de instantâneos pois caso precise, repetir o processo de coleta
pode não ser possível. Um exemplo é editar um disco virtual que estava atrelado a uma máquina virtual da qual se gerou os instantâneos, tal ação pode levar a perda de dados.

Como métrica, o autor relaciona o tamanho da memória alocada na máquina virtual com o tempo necessário para gerar o instantâneo de acordo com a tabela 1 abaixo

\begin{table}[h!]
\centering
\caption{Memória alocada X Tempo de captação}
\label{my-label}
\begin{tabular}{c|c}
\hline
\textbf{Memória alocada na VM} & \textbf{Tempo geração snapshot} \\ \hline
512 Mb                         & 15 segundos                     \\ \hline
1 Gb                           & 20 segundos                     \\ \hline
4 Gb                           & 36 segundos                     \\ \hline
\end{tabular}
\end{table}

Quando comparado a este trabalho, a presente proposta tem por vantagens coletar apenas a informações de memória e usar a janela de coleta de \textbf{x} dias antes do ataque para 
manter sob controle a quantidade de informação que precisa ser analisada. Tomando como referência a tabela acima, conseguiremos um menor tempo de coleta da informação de 
memória pertinente a investigação, permitindo um menor espaço de tempo entre as coletas, gerando menos impacto na performance da aplicação e mais dados para a investigação.
Propondo a autilização de container para associar a evidência a sua origem, tornamos o processo independente de máquina. \\

\item \textbf{A log based approach to make digital forensics easier on cloud computing \cite{Sang2013} }: Método sugere salvar a informação coletada fora da núvem de modo 
a driblar os problemas de multi-inquilinato e multi-jurisdição, usa um mecanismo de hash para garantir a autenticidade e integridade da informação mas não dá detalhes da 
implementação e não descreve como controla o acesso a evidência armazenada. Segundo o próprio autor, o método não funciona em IaaS. Precisa da cooperação do provedor de 
nuvem pois depende das informações que este último decidiu adicionar ao log. O método não é aplicável a coleta de informações de memória.

A proposta não coleta dados de memória por decisão do autor, esta proposta entrou na lista pela abordagem baseada em log. Neste quesito, a presente proposta é a melhor pois 
garante que a evidência não foi alterada ou destruida no transporte e o acesso a mesma é controlado. No âmbito da informação coletada, a presente proposta não depende das 
decisões do provedor de nuvem sobre o que guardar no log para conseguir a evidência. \\

\item \textbf{Volatile memory acquisition using backup for forensic investigation \cite{Dezfouli2012} }: Técnica desenvolvida para dispositivos móveis que sugere a utilização
do próprio como repositório das evidências coletadas da memória. Para manter a utilização de espaço ao mínimo sugere manter apenas o último estado conhecido da memória.
 
É uma técnica interessante do ponto de vista de estratégia de armazenamento quando guarda apenas o último estado da memória. Essa abordagem porém perde a informação do momento 
do ataque e não consegue descrever o sistema antes do mesmo. Do resto da proposta não é aplicável para este projeto pois, armazenando a evidência na máquina a mesma seria perdida
quando a máquina fosse despejada do pool e seus recursos liberados. A cadeia de custódia não é abordada na proposta.\\

\item \textbf{Narrowing the semantic gap in virtual machine introspection \cite{Dolan-Gavitt2011a} }: Esta proposta é uma combinação da técnica de introspecção em máquina virtual
e a integração com a API do hypervisor. O principal objetivo é diminuir o gap semântico para facilitar a análise da evidência. Para isso o autor implementa um API para transformar
dados de baixo nível em informação de alto nível. Depende de cooperação do provedor de serviço de nuvem, não tem conhecimento da máquina hospedada e não vai além da coleta, não 
descreve como resolve a cadeia de custódia. Tem a vantagem de ser imune a técnicas anti-forenses.

Como métrica o autor relaciona o tamanho médio do trace gerado a partir da uma evidência de memória coletada de alguns processos em vários sistemas operacionais de acordo 
com a tabela 2 abaixo. Essa métrica pode se relacionar a presente proposta como o volume de informação extraida de uma evidência.

\begin{table}[]
\centering
\caption{Tamanho do trace coletado de vários programas}
\label{my-label}
\begin{tabular}{l|c|c}
\hline
\textbf{Sistema Operacional}		&\textbf{Programa}	      & \textbf{Tamanho do Trace}\\ \hline
\multirow{6}{*}{Windows}                & getpid                      & 3549			 \\
                                        & gettime                     & 7715    		 \\
                                        & pslist                      & 302082  		 \\
                                        & lsmod                       & 195488   		 \\
                                        & getpsfile                   & 49588   		 \\
                                        & getdrvfile                  & 194765  		 \\
\multirow{5}{*}{Linux}                  & getpid                      & 133047                   \\ \hline
                                        & gettime                     & 75074                    \\
                                        & pslist                      & 6107214                  \\
                                        & lsmod                       & 1936439                  \\
                                        & getpsfile                   & 14752561                 \\
\multirow{6}{*}{Haiku}                  & getpid                      & 18242                    \\ \hline
                                        & gettime                     & 9982                     \\
                                        & pslist                      & 362127                   \\
                                        & lsmod                       & 850363                   \\
                                        & getpsfile                   & 249663                   \\
                                        & getdrvfile                  & 522299                   \\ 
\end{tabular}
\end{table}

Quando comparado a este trabalho, a presente proposta tem por vantagens ser arcabouço de coleta e armazenamento de evidências não apenas um método de coleta. Empregando a
estratégia de realizar a coleta diretamente na máquina não tem o problema do gap semântico próprio das soluções baseadas em introspecção. Como conhece o contexto do que 
está rodando dentro da máquina virtual podemos direcionar a coleta de modo que seja mais eficiênte. O autor usa outras métricas voltadas ao processamento da evidência antes
de sua análise para diminuir a quantidade de informação a se analisar mas não dá detalhes de como esse processamento.\\

\item \textbf{Digital Forensics as a Service: A game changer \cite{VanBaar2014} }: Esta proposta é focada em uma mudança no armazenamento e forma de trabalho dos peritos forenses.
Propõe que a forense seja oferecida como um serviço e que as evidências sejam armazenadas em um local centralizado com o devido controle de acesso e garantia de integridade da 
evidência. Descreve a arquitetura de armazenamento, qual o perfil que deve ter acesso a evidência e como é este acesso. Esta proposta não é focada apenas em incidentes na nuvem
mas em qualquer outro incidente.

Embora seja uma ótima proposta de armazenamento de evidências e controle de acesso a elas, ele não descreve o processo de coleta nem de transporte. É uma proposta focada 
mais na solução de problemas relacionados a manipulação da informação após a coleta, transporte e armazenamento. Não toca no assunto de coleta qualquer que seja, em nuvem
ou fisica.\\

\item \textbf{Live Digital Forensics in a Virtual Machine \cite{Zhang2010} }: Proposta para coletar informações de memória de máquinas virtuais através de instantâneos das mesmas.
O metodo de coleta envolve tomar o instantâneo da máquina, no diretório onde o mesmo foi armazenado pegar o arquivo referente a memória e abri-lo / analisá-lo usando um programa de
leitura de memória de mercado. O autor não trás detalhes do transporte, armazenamento ou controle de acesso. Precisa que a máquina exista para conseguir coletar a informação e 
o processo é dependente de intervenção humana. Analisando com mais cuidado é possível repetir a coleta mesmo sem a máquina existir uma vez que temos o instantâneo mas o autor não
dá detalhes do caso.

Quando comparado a este trabalho, a presente proposta tem por vantagens a menor quantidade de informação necessária à investigação através da implementação da janela de x
dias antes do incidente. Como o processo é automático, uma vez disparado não requer intervenção humana. A presente proposta descreve como garante a cadeia de custódia da 
evidência e consigue reproduzir o processo de coleta mesmo se a máquina não existir  mais pois a evidência está atrelada ao container. \\

\item \textbf{Comparative Analisys of volatile memory forensics \cite{Aljaedi2011} }: Levantamento sobre o impacto da realização de forense de memória ao vivo nas máquinas
virtuais na nuvem. Mostra que a quantidade de informações referentes a processos não alocados e página de memória perdida é um ponto a considerar quando se inicia uma
ferramente para análise de memória em uma máquina já funcionando como.

Como métrica o autor relaciona a porcentagem média de processos que permanecem na memória antes e depois que a ferramenta de coleta foi disparada com a quantidade de memória
alocada na VM de acordo com a tabela 3 abaixo.

\begin{table}[h!]
\centering
\caption{Média de processos que permaneceram na memória}
\label{my-label}
\begin{tabular}{c|c|c}
\hline
\textbf{Memória alocada a VM} & \textbf{\% antes ativação} 			  & \textbf{\% depois ativação} \\ \hline
1 Gb                          & 78,33                                             & 61,66                       \\ \hline
512 Mb                        & 73,33                                             & 46,66                       \\ \hline
256 Mb                        & 50,55                                             & 35,00                                                          
\end{tabular}
\end{table}

Em outra métrica o autor relaciona a porcentagem média de páginas de memória alterada nos métodos de \textbf{\% análise ao vivo}, onde a coleta é realizada com o sistema rodando e 
\textbf{\% cópia de memória} onde a máquina virtual é pausada para a realização de coleta de informação com a quantidade de memória alocada de acordo com a tabela abaixo. Em ambos
os casos as duas métricas são comparações entre o estado da memória antes e depois da inicialização da ferramenta de coleta.

\begin{table}[h!]
\centering
\caption{Média de página de memória alterada}
\label{my-label}
\begin{tabular}{c|c|c}
\hline
\textbf{Memória alocada a VM} & \textbf{\% análise ao vivo} 			  & \textbf{\% cópia de memória} \\ \hline
1 Gb                          & 7,99                                              & 5,95                         \\ \hline
512 Mb                        & 32,53                                             & 8,75                         \\ \hline
256 Mb                        & 52,37                                             & 25,46                                                         
\end{tabular}
\end{table}

Quando comparado a este trabalho a presente proposta tem por vantagem estar sempre em execução, sujeitando-se aos efeitos de escalonamento do processo e 
gerenciamento de páginas de memória pelo sistema operacional, assim não gera as perdas de informação de processos não alocados e páginas de memória referentes ao 
rearranjo que o kernel faz quando uma nova aplicação é iniciada.

\end{itemize}

\textbf{Tabela comparativa das soluções}

\begin{table}[h!]
\centering
\caption{Comparativo de soluções}
\label{my-label}
\begin{tabular}{L|L|L|L|L|L|L}
\hline
\textbf{}			& \rotatebox{90}{\textbf{Coleta  é continua?}}      & \rotatebox{90}{\textbf{Precisa de tradução de endereços para análise?}}      & \rotatebox{90}{\textbf{É independente de VM?}}      & \rotatebox{90}{\textbf{Conhece o que esta rodando na VM?}}      & \rotatebox{90}{\textbf{Garante cadeia de custódia?}}      & \rotatebox{90}{\textbf{Preserva evidência de memória volátil?}}      \\ \hline
\cite{George2012}		& Não                                               & Não                                                                          & Não                                                 & Sim                                                             & Não                                                       & Não                                                                  \\ \hline
\cite{Poisel2013}		& Não                                               & Sim                                                                          & Não                                                 & Não                                                             & Não                                                       & Sim                                                                  \\ \hline
\cite{Dykstra2013}		& Não                                               & Não                                                                          & Não                                                 & Não                                                             & Não                                                       & Não                                                                  \\ \hline
\cite{Do2014}			& Não                                               & Não                                                                          & Não                                                 & Não                                                             & Não                                                       & Não                                                                  \\ \hline
\cite{Reichert2015}		& Não                                               & Não                                                                          & Não                                                 & Sim                                                             & Sim                                                       & Não                                                                  \\ \hline
\cite{Sang2013}			& Sim                                               & Não                                                                          & Sim                                                 & Sim                                                             & Não                                                       & Não                                                                  \\ \hline
\cite{Dolan-Gavitt2011a}	& Não                                               & Não                                                                          & Não                                                 & Não                                                             & Não                                                       & Sim                                                                  \\ \hline
\cite{Aljaedi2011}		& Não                                               & Não                                                                          & Não                                                 & Sim                                                             & Não                                                       & Sim                                                                  \\ \hline
\cite{Dezfouli2012}		& Sim                                               & Não                                                                          & Não                                                 & Sim                                                             & Não                                                       & Sim                                                                  \\ \hline
\cite{VanBaar2014}		& Sim                                               & Não                                                                          & Não                                                 & Não                                                             & Sim                                                       & Não                                                                  \\
\end{tabular}
\end{table}

\section{Solução proposta}

\subsection{Objetivos}

Lorem ipsum dolor sit amet, consectetuer adipiscing elit. Aenean commodo ligula eget dolor. Aenean massa. Cum sociis natoque penatibus et magnis dis parturient montes, 
nascetur ridiculus mus. Donec quam felis, ultricies nec, pellentesque eu, pretium quis, sem. Nulla consequat massa quis enim. Donec pede justo, fringilla vel, aliquet 
nec, vulputate eget, arcu. In enim justo, rhoncus ut, imperdiet a, venenatis vitae, justo. Nullam dictum felis eu pede mollis pretium. Integer tincidunt. Cras dapibus.
Vivamus elementum semper nisi. Aenean vulputate eleifend tellus. Aenean leo ligula, porttitor eu, consequat vitae, eleifend ac, enim. Aliquam lorem ante, dapibus in, 
viverra quis, feugiat a, tellus. Phasellus viverra nulla ut metus varius laoreet. Quisque rutrum. Aenean imperdiet. Etiam ultricies nisi vel augue. Curabitur ullamcorper 
ultricies nisi. Nam eget dui. Etiam rhoncus. Maecenas tempus, tellus eget condimentum rhoncus, sem quam semper libero, sit amet adipiscing sem neque sed ipsum. Nam quam 
nunc, blandit vel, luctus pulvinar, hendrerit id, lorem. Maecenas nec odio et ante tincidunt tempus. Donec vitae sapien ut libero venenatis faucibus. Nullam quis ante. 
Etiam sit amet orci eget eros faucibus tincidunt. Duis leo. Sed fringilla mauris sit amet nibh. Donec sodales sagittis magna. Sed consequat, leo eget bibendum sodales, 
augue velit cursus nunc, 

I wish you the best of success.

\hfill mds
 
\hfill December 27, 2012

\subsection{Descrição}

Lorem ipsum dolor sit amet, consectetuer adipiscing elit. Aenean commodo ligula eget dolor. Aenean massa. Cum sociis natoque penatibus et magnis dis parturient montes, 
nascetur ridiculus mus. Donec quam felis, ultricies nec, pellentesque eu, pretium quis, sem. Nulla consequat massa quis enim. Donec pede justo, fringilla vel, aliquet 
nec, vulputate eget, arcu. In enim justo, rhoncus ut, imperdiet a, venenatis vitae, justo. Nullam dictum felis eu pede mollis pretium. Integer tincidunt. Cras dapibus.
Vivamus elementum semper nisi. Aenean vulputate eleifend tellus. Aenean leo ligula, porttitor eu, consequat vitae, eleifend ac, enim. Aliquam lorem ante, dapibus in, 
viverra quis, feugiat a, tellus. Phasellus viverra nulla ut metus varius laoreet. Quisque rutrum. Aenean imperdiet. Etiam ultricies nisi vel augue. Curabitur ullamcorper 
ultricies nisi. Nam eget dui. Etiam rhoncus. Maecenas tempus, tellus eget condimentum rhoncus, sem quam semper libero, sit amet adipiscing sem neque sed ipsum. Nam quam 
nunc, blandit vel, luctus pulvinar, hendrerit id, lorem. Maecenas nec odio et ante tincidunt tempus. Donec vitae sapien ut libero venenatis faucibus. Nullam quis ante. 
Etiam sit amet orci eget eros faucibus tincidunt. Duis leo. Sed fringilla mauris sit amet nibh. Donec sodales sagittis magna. Sed consequat, leo eget bibendum sodales, 
augue velit cursus nunc, 

I wish you the best of success.

\hfill mds
 
\hfill December 27, 2012

\subsection{Limitações}

Lorem ipsum dolor sit amet, consectetuer adipiscing elit. Aenean commodo ligula eget dolor. Aenean massa. Cum sociis natoque penatibus et magnis dis parturient montes, 
nascetur ridiculus mus. Donec quam felis, ultricies nec, pellentesque eu, pretium quis, sem. Nulla consequat massa quis enim. Donec pede justo, fringilla vel, aliquet 
nec, vulputate eget, arcu. In enim justo, rhoncus ut, imperdiet a, venenatis vitae, justo. Nullam dictum felis eu pede mollis pretium. Integer tincidunt. Cras dapibus.
Vivamus elementum semper nisi. Aenean vulputate eleifend tellus. Aenean leo ligula, porttitor eu, consequat vitae, eleifend ac, enim. Aliquam lorem ante, dapibus in, 
viverra quis, feugiat a, tellus. Phasellus viverra nulla ut metus varius laoreet. Quisque rutrum. Aenean imperdiet. Etiam ultricies nisi vel augue. Curabitur ullamcorper 
ultricies nisi. Nam eget dui. Etiam rhoncus. Maecenas tempus, tellus eget condimentum rhoncus, sem quam semper libero, sit amet adipiscing sem neque sed ipsum. Nam quam 
nunc, blandit vel, luctus pulvinar, hendrerit id, lorem. Maecenas nec odio et ante tincidunt tempus. Donec vitae sapien ut libero venenatis faucibus. Nullam quis ante. 
Etiam sit amet orci eget eros faucibus tincidunt. Duis leo. Sed fringilla mauris sit amet nibh. Donec sodales sagittis magna. Sed consequat, leo eget bibendum sodales, 
augue velit cursus nunc, 

I wish you the best of success.

\hfill mds
 
\hfill December 27, 2012

\subsection{Trabalhos futuros}

Lorem ipsum dolor sit amet, consectetuer adipiscing elit. Aenean commodo ligula eget dolor. Aenean massa. Cum sociis natoque penatibus et magnis dis parturient montes, 
nascetur ridiculus mus. Donec quam felis, ultricies nec, pellentesque eu, pretium quis, sem. Nulla consequat massa quis enim. Donec pede justo, fringilla vel, aliquet 
nec, vulputate eget, arcu. In enim justo, rhoncus ut, imperdiet a, venenatis vitae, justo. Nullam dictum felis eu pede mollis pretium. Integer tincidunt. Cras dapibus.
Vivamus elementum semper nisi. Aenean vulputate eleifend tellus. Aenean leo ligula, porttitor eu, consequat vitae, eleifend ac, enim. Aliquam lorem ante, dapibus in, 
viverra quis, feugiat a, tellus. Phasellus viverra nulla ut metus varius laoreet. Quisque rutrum. Aenean imperdiet. Etiam ultricies nisi vel augue. Curabitur ullamcorper 
ultricies nisi. Nam eget dui. Etiam rhoncus. Maecenas tempus, tellus eget condimentum rhoncus, sem quam semper libero, sit amet adipiscing sem neque sed ipsum. Nam quam 
nunc, blandit vel, luctus pulvinar, hendrerit id, lorem. Maecenas nec odio et ante tincidunt tempus. Donec vitae sapien ut libero venenatis faucibus. Nullam quis ante. 
Etiam sit amet orci eget eros faucibus tincidunt. Duis leo. Sed fringilla mauris sit amet nibh. Donec sodales sagittis magna. Sed consequat, leo eget bibendum sodales, 
augue velit cursus nunc, 

I wish you the best of success.

\hfill mds
 
\hfill December 27, 2012

% An example of a floating figure using the graphicx package.
% Note that \label must occur AFTER (or within) \caption.
% For figures, \caption should occur after the \includegraphics.
% Note that IEEEtran v1.7 and later has special internal code that
% is designed to preserve the operation of \label within \caption
% even when the captionsoff option is in effect. However, because
% of issues like this, it may be the safest practice to put all your
% \label just after \caption rather than within \caption{}.
%
% Reminder: the "draftcls" or "draftclsnofoot", not "draft", class
% option should be used if it is desired that the figures are to be
% displayed while in draft mode.
%
%\begin{figure}[!t]
%\centering
%\includegraphics[width=2.5in]{myfigure}
% where an .eps filename suffix will be assumed under latex, 
% and a .pdf suffix will be assumed for pdflatex; or what has been declared
% via \DeclareGraphicsExtensions.
%\caption{Simulation Results.}
%\label{fig_sim}
%\end{figure}

% Note that IEEE typically puts floats only at the top, even when this
% results in a large percentage of a column being occupied by floats.


% An example of a double column floating figure using two subfigures.
% (The subfig.sty package must be loaded for this to work.)
% The subfigure \label commands are set within each subfloat command,
% and the \label for the overall figure must come after \caption.
% \hfil is used as a separator to get equal spacing.
% Watch out that the combined width of all the subfigures on a 
% line do not exceed the text width or a line break will occur.
%
%\begin{figure*}[!t]
%\centering
%\subfloat[Case I]{\includegraphics[width=2.5in]{box}%
%\label{fig_first_case}}
%\hfil
%\subfloat[Case II]{\includegraphics[width=2.5in]{box}%
%\label{fig_second_case}}
%\caption{Simulation results.}
%\label{fig_sim}
%\end{figure*}
%
% Note that often IEEE papers with subfigures do not employ subfigure
% captions (using the optional argument to \subfloat[]), but instead will
% reference/describe all of them (a), (b), etc., within the main caption.


% An example of a floating table. Note that, for IEEE style tables, the 
% \caption command should come BEFORE the table. Table text will default to
% \footnotesize as IEEE normally uses this smaller font for tables.
% The \label must come after \caption as always.
%
%\begin{table}[!t]
%% increase table row spacing, adjust to taste
%\renewcommand{\arraystretch}{1.3}
% if using array.sty, it might be a good idea to tweak the value of
% \extrarowheight as needed to properly center the text within the cells
%\caption{An Example of a Table}
%\label{table_example}
%\centering
%% Some packages, such as MDW tools, offer better commands for making tables
%% than the plain LaTeX2e tabular which is used here.
%\begin{tabular}{|c||c|}
%\hline
%One & Two\\
%\hline
%Three & Four\\
%\hline
%\end{tabular}
%\end{table}


% Note that IEEE does not put floats in the very first column - or typically
% anywhere on the first page for that matter. Also, in-text middle ("here")
% positioning is not used. Most IEEE journals/conferences use top floats
% exclusively. Note that, LaTeX2e, unlike IEEE journals/conferences, places
% footnotes above bottom floats. This can be corrected via the \fnbelowfloat
% command of the stfloats package.

\section{Conclusion}

Lorem ipsum dolor sit amet, consectetuer adipiscing elit. Aenean commodo ligula eget dolor. Aenean massa. Cum sociis natoque penatibus et magnis dis parturient montes, 
nascetur ridiculus mus. Donec quam felis, ultricies nec, pellentesque eu, pretium quis, sem. Nulla consequat massa quis enim. Donec pede justo, fringilla vel, aliquet 
nec, vulputate eget, arcu. In enim justo, rhoncus ut, imperdiet a, venenatis vitae, justo. Nullam dictum felis eu pede mollis pretium. Integer tincidunt. Cras dapibus.
Vivamus elementum semper nisi. Aenean vulputate eleifend tellus. Aenean leo ligula, porttitor eu, consequat vitae, eleifend ac, enim. Aliquam lorem ante, dapibus in, 
viverra quis, feugiat a, tellus. Phasellus viverra nulla ut metus varius laoreet. Quisque rutrum. Aenean imperdiet. Etiam ultricies nisi vel augue. Curabitur ullamcorper 
ultricies nisi. Nam eget dui. Etiam rhoncus. Maecenas tempus, tellus eget condimentum rhoncus, sem quam semper libero, sit amet adipiscing sem neque sed ipsum. Nam quam 
nunc, blandit vel, luctus pulvinar, hendrerit id, lorem. Maecenas nec odio et ante tincidunt tempus. Donec vitae sapien ut libero venenatis faucibus. Nullam quis ante. 
Etiam sit amet orci eget eros faucibus tincidunt. Duis leo. Sed fringilla mauris sit amet nibh. Donec sodales sagittis magna. Sed consequat, leo eget bibendum sodales, 
augue velit cursus nunc, 

% conference papers do not normally have an appendix


% use section* for acknowledgement
\section*{Acknowledgment}

Lorem ipsum dolor sit amet, consectetuer adipiscing elit. Aenean commodo ligula eget dolor. Aenean massa. Cum sociis natoque penatibus et magnis dis parturient montes, 
nascetur ridiculus mus. Donec quam felis, ultricies nec, pellentesque eu, pretium quis, sem. Nulla consequat massa quis enim. Donec pede justo, fringilla vel, aliquet 
nec, vulputate eget, arcu. In enim justo, rhoncus ut, imperdiet a, venenatis vitae, justo. Nullam dictum felis eu pede mollis pretium. Integer tincidunt. Cras dapibus.
Vivamus elementum semper nisi. Aenean vulputate eleifend tellus. Aenean leo ligula, porttitor eu, consequat vitae, eleifend ac, enim. Aliquam lorem ante, dapibus in, 
viverra quis, feugiat a, tellus. Phasellus viverra nulla ut metus varius laoreet. Quisque rutrum. Aenean imperdiet. Etiam ultricies nisi vel augue. Curabitur ullamcorper 
ultricies nisi. Nam eget dui. Etiam rhoncus. Maecenas tempus, tellus eget condimentum rhoncus, sem quam semper libero, sit amet adipiscing sem neque sed ipsum. Nam quam 
nunc, blandit vel, luctus pulvinar, hendrerit id, lorem. Maecenas nec odio et ante tincidunt tempus. Donec vitae sapien ut libero venenatis faucibus. Nullam quis ante. 
Etiam sit amet orci eget eros faucibus tincidunt. Duis leo. Sed fringilla mauris sit amet nibh. Donec sodales sagittis magna. Sed consequat, leo eget bibendum sodales, 
augue velit cursus nunc, 


% trigger a \newpage just before the given reference
% number - used to balance the columns on the last page
% adjust value as needed - may need to be readjusted if
% the document is modified later
%\IEEEtriggeratref{8}
% The "triggered" command can be changed if desired:
%\IEEEtriggercmd{\enlargethispage{-5in}}

% references section

% can use a bibliography generated by BibTeX as a .bbl file
% BibTeX documentation can be easily obtained at:
% http://www.ctan.org/tex-archive/biblio/bibtex/contrib/doc/
% The IEEEtran BibTeX style support page is at:
% http://www.michaelshell.org/tex/ieeetran/bibtex/
%\bibliographystyle{IEEEtran}
% argument is your BibTeX string definitions and bibliography database(s)
%\bibliography{IEEEabrv,../bib/paper}
%
% <OR> manually copy in the resultant .bbl file
% set second argument of \begin to the number of references
% (used to reserve space for the reference number labels box)
\begin{thebibliography}{1}
  
\bibitem{George2012}
GEORGE, S.; VENTER, H.; THOMAS, F. {Digital Forensic Framework for a Cloud
  Environment. In:  CUNNINGHAM, P.; CUNNINGHAM, M. (Ed.). \emph{IST Africa
  2012}. Tanzania: Internation Information Management Corporation, 2012.
  p.~1--8.
  ISBN 9781905824342.}
  
\bibitem{Poisel2013}
{POISEL, R.; MALZER, E.; TJOA, S. {Evidence and cloud computing: The virtual
  machine introspection approach}.
\emph{Journal of Wireless Mobile Networks, Ubiquitous Computing, and Dependable
  Applications}, v.~4, n.~1, p. 135--152, 2013.
ISSN 20935374 (ISSN).}

\bibitem{Dykstra2013}
{DYKSTRA, J.; SHERMAN, A.~T. {Design and implementation of FROST: Digital
  forensic tools for the OpenStack cloud computing platform}.
\emph{Digital Investigation}, Elsevier Ltd, v.~10, n. SUPPL., p. S87--S95,
  2013.
ISSN 17422876.}

\bibitem{Reichert2015}
{REICHERT, Z.; RICHARDS, K.; YOSHIGOE, K. {Automated forensic data acquisition
  in the cloud}.
\emph{Proceedings - 11th IEEE International Conference on Mobile Ad Hoc and
  Sensor Systems, MASS 2014}, p. 725--730, 2015.}

\bibitem{Sang2013}
{SANG, T. {A log-based approach to make digital forensics easier on cloud
  computing}.
\emph{Proceedings of the 2013 3rd International Conference on Intelligent
  System Design and Engineering Applications, ISDEA 2013}, p. 91--94, 2013.}
  
\bibitem{Dezfouli2012}
{DEZFOULI, F.~N. et al. {Volatile memory acquisition using backup for forensic
  investigation}.
\emph{Proceedings 2012 International Conference on Cyber Security, Cyber
  Warfare and Digital Forensic, CyberSec 2012}, p. 186--189, 2012.}
  
\bibitem{Dolan-Gavitt2011a}
{DOLAN-GAVITT, B. et al. {Virtuoso: Narrowing the semantic gap in virtual
  machine introspection}.
\emph{Proceedings - IEEE Symposium on Security and Privacy}, p. 297--312, 2011.
ISSN 10816011.}

\bibitem{VanBaar2014}
{BAAR, R.~B. van; BEEK, H. M.~A. van; EIJK, E.~J. van. {Digital Forensics as a
  Service: A game changer}.
\emph{Digital Investigation}, Elsevier Ltd, v.~11, p. S54--S62, 2014.
ISSN 17422876.}

\bibitem{Zhang2010}
{ZHANG, L.; ZHANG, D.; WANG, L. {Live Digital Forensics in a Virtual Machine}.
  In:  \emph{2010 Internation Conference on Computer Application and System
  Modelling (ICCASM 2010)}. [S.l.: s.n.], 2010. v.~6, p. 328--332.}
  
\bibitem{Aljaedi2011}
{ALJAEDI, A. et al. {Comparative Analysis of Volatile Memory Forensics}.
\emph{IEEE International Conference on Privacy, Security, Risk and Trust
  (PASSAT) and IEEE International Conference on Social Computing (SocialCom)},
  p. 1253--1258, 2011.}
  
\bibitem{Do2014}
{BARBARA, D. \emph{{Desafios da per{\'{i}}cia forense em um ambiente de
  computa{\c{c}}{\~{a}}o nas nuvens}}.
[S.l.], 2014.}



\end{thebibliography}

% that's all folks
\end{document}


