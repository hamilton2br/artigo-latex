%\title{Modelo de Projeto de pesquisa}
%% abtex2-modelo-projeto-pesquisa.tex, v-1.9 laurocesar
%% Copyright 2012-2013 by abnTeX2 group at http://abntex2.googlecode.com/ 
%%
%% This work may be distributed and/or modified under the
%% conditions of the LaTeX Project Public License, either version 1.3
%% of this license or (at your option) any later version.
%% The latest version of this license is in
%%   http://www.latex-project.org/lppl.txt
%% and version 1.3 or later is part of all distributions of LaTeX
%% version 2005/12/01 or later.
%%
%% This work has the LPPL maintenance status `maintained'.
%% 
%% The Current Maintainer of this work is the abnTeX2 team, led
%% by Lauro César Araujo. Further information are available on 
%% http://abntex2.googlecode.com/
%%
%% This work consists of the files abntex2-modelo-projeto-pesquisa.tex
%% and abntex2-modelo-references.bib
%%

% ------------------------------------------------------------------------
% ------------------------------------------------------------------------
% abnTeX2: Modelo de Projeto de pesquisa em conformidade com 
% ABNT NBR 15287:2011 Informação e documentação - Projeto de pesquisa -
% Apresentação 
% ------------------------------------------------------------------------ 
% ------------------------------------------------------------------------

\documentclass[
	% -- opções da classe memoir --
	12pt,				% tamanho da fonte
	openright,			% capítulos começam em pág ímpar (insere página vazia caso preciso)
	oneside,			% para impressão em verso e anverso. Oposto a oneside
	a4paper,			% tamanho do papel. 
	% -- opções da classe abntex2 --
	%chapter=TITLE,		% títulos de capítulos convertidos em letras maiúsculas
	%section=TITLE,		% títulos de seções convertidos em letras maiúsculas
	%subsection=TITLE,	% títulos de subseções convertidos em letras maiúsculas
	%subsubsection=TITLE,% títulos de subsubseções convertidos em letras maiúsculas
	% -- opções do pacote babel --
	english,			% idioma adicional para hifenização
	french,				% idioma adicional para hifenização
	spanish,			% idioma adicional para hifenização
	brazil,				% o último idioma é o principal do documento
	]{abntex2}

% ---
% PACOTES
% ---

% ---
% Pacotes fundamentais 
% ---
\usepackage{lmodern}			% Usa a fonte Latin Modern
\usepackage[T1]{fontenc}		% Selecao de codigos de fonte.
\usepackage[utf8]{inputenc}		% Codificacao do documento (conversão automática dos acentos)
\usepackage{indentfirst}		% Indenta o primeiro parágrafo de cada seção.
\usepackage{color}				% Controle das cores
\usepackage{graphicx}			% Inclusão de gráficos
\usepackage{microtype} 			% para melhorias de justificação
% ---

% ---
% Pacotes adicionais, usados apenas no âmbito do Modelo Canônico do abnteX2
% ---
\usepackage{lipsum}				% para geração de dummy text
% ---

% ---
% Pacotes de citações
% ---
\usepackage[brazilian,hyperpageref]{backref}	 % Paginas com as citações na bibl
\usepackage[alf]{abntex2cite}	% Citações padrão ABNT

% --- 
% CONFIGURAÇÕES DE PACOTES
% --- 

% ---
% Configurações do pacote backref
% Usado sem a opção hyperpageref de backref
\renewcommand{\backrefpagesname}{Citado na(s) página(s):~}
% Texto padrão antes do número das páginas
\renewcommand{\backref}{}
% Define os textos da citação
\renewcommand*{\backrefalt}[4]{
	\ifcase #1 %
		Nenhuma citação no texto.%
	\or
		Citado na página #2.%
	\else
		Citado #1 vezes nas páginas #2.%
	\fi}%
% ---

% ---
% Informações de dados para CAPA e FOLHA DE ROSTO
% ---
\titulo{Coletando dados de memória de uma máquina em nuvem para análise forense}
\autor{Hamilton Fonte II}
\orientador{Marcos Antonio Simplício Jr}
\local{São Paulo, Brasil}
\data{2016, v-0.1}
\instituicao{%
  Universidade de São Paulo -- USP
  \par
  Escola Politécnica - Engenharia de Computação
  \par
  Programa de Pós Graduação em Engenharia Elétrica - Mestrado}
\tipotrabalho{Plano de Pesquisa de Pós-Graduação - Mestrado}
% O preambulo deve conter o tipo do trabalho, o objetivo, 
% o nome da instituição e a área de concentração 
\preambulo{Projeto de pesquisa para a disciplina Metodolodia de Pesquisa 
Científica em Engenharia de Computação.}
% ---

% ---
% Configurações de aparência do PDF final

% alterando o aspecto da cor azul
\definecolor{blue}{RGB}{41,5,195}

% informações do PDF
\makeatletter
\hypersetup{
     	%pagebackref=true,
		pdftitle={\@title}, 
		pdfauthor={\@author},
    	pdfsubject={\imprimirpreambulo},
	    pdfcreator={LaTeX with abnTeX2},
		pdfkeywords={abnt}{latex}{abntex}{abntex2}{projeto de pesquisa}, 
		colorlinks=true,       		% false: boxed links; true: colored links
    	linkcolor=blue,          	% color of internal links
    	citecolor=blue,        		% color of links to bibliography
    	filecolor=magenta,      		% color of file links
		urlcolor=blue,
		bookmarksdepth=4
}
\makeatother
% --- 

% --- 
% Espaçamentos entre linhas e parágrafos 
% --- 

% O tamanho do parágrafo é dado por:
\setlength{\parindent}{1.3cm}

% Controle do espaçamento entre um parágrafo e outro:
\setlength{\parskip}{0.2cm}  % tente também \onelineskip

% ---
% compila o indice
% ---
\makeindex
% ---

% ----
% Início do documento
% ----
\begin{document}

% Retira espaço extra obsoleto entre as frases.
\frenchspacing 

% ----------------------------------------------------------
% ELEMENTOS PRÉ-TEXTUAIS
% ----------------------------------------------------------
% \pretextual

% ---
% Capa
% ---
\imprimircapa
% ---

% ---
% Folha de rosto
% ---
\imprimirfolhaderosto
% ---

% ---
% NOTA DA ABNT NBR 15287:2011, p. 4:
%  ``Se exigido pela entidade, apresentar os dados curriculares do autor em
%     folha ou página distinta após a folha de rosto.''
% ---

% ---
% inserir o sumario
% ---
\tableofcontents
\cleardoublepage
% ---


% ----------------------------------------------------------
% ELEMENTOS TEXTUAIS
% ----------------------------------------------------------
\textual

% ----------------------------------------------------------
% Introdução
% ----------------------------------------------------------
\chapter{Introdução}

Aqui vai a introdução

% ----------------------------------------------------------
% Capitulo de justificativa 
% ----------------------------------------------------------
\chapter{Justificativa}

Aqui vai a justificativa

% ----------------------------------------------------------
% Capitulo de Objetivos 
% ----------------------------------------------------------
\chapter{Objetivos}

Aqui vão os objetivos

% ----------------------------------------------------------
% Plano de Trabalho 
% ----------------------------------------------------------
\chapter{Métodos}

Aqui vão os métodos

% ----------------------------------------------------------
% Material e métodos
% ----------------------------------------------------------
\chapter{Revisão Bibliográfica}

\begin{itemize}

\item \textbf{Digital forensics framework for a cloud environment \cite{George2012} }: Framework para coleta de dados de máquinas virtuais. Tem duas formas de acionamento,
a manual e outra automaticamente, integrada com algum sistema de detecção de ameaça. Quando acionado, escuta a rede, determina qual a máquina é objeto de investigação e 
coleta informações de log e tráfego de rede e associa ao usuário das respectivas máquinas. Propõe o armazenamento das evidências em local fora da núvem para escapar de 
problemas de jurisdição e multi-inquilino mas tem inteligência para usar a própria nuvem como armazenamento caso o espaço fora da nuvem acabe.

A proposta dá a entender que é aplicável apenas a um sistema virtual estático, onde o número e organização das máquinas é constante. De informação volátil, coleta apenas 
tráfego de rede, não coleta memória. Com a forma de acionamento descrito ele não consegue descrever, com as evidências, como era o sistema antes do ataque. Apesar de armazenar 
a evidëncia fora da nuvem, não da detalhes de cadeia de custódia, garantia de integridade e confidencialidade. O autor não descreve como garante que a evidência não foi alterada
destruída no transporte até o local de armazenamento nem como controla o acesso a evidência.

Quando comparado a este trabalho, a presente proposta tem por vantagens a utilização de container para associar a evidência a sua origem tornando o processo independente 
de máquina. Como a implementação de uma janela de x dias de coleta antes da detecção do ataque é possível descrever, através de evidência, como era o sistema antes do mesmo. 
Com isso a solução apresentada consegue evidências em um cenário de infra-estrutura dinâmica. São tomadas precauções para garantir a integridade e confidencialidade dos 
dados transportando-os via TLS os para um local fora da nuvem e controlando o acesso aos mesmos.\\
 
\item \textbf{Evidence and cloud computing the virtual machine introspection approach \cite{Poisel2013} }: Descreve um método de coleta de informações de máquinas
em nuvem através da técnica de introspecção em máquina virtual, onde se acessa os dados das máquinas virtuais através do hypervisor. Propõe que o processo seja disparado
automaticamente por demanda integrado a um sistema de detecção de ameaça mas também suporta acionamento manual.

A técnica descrita cobre apenas o processo de coleta de informações, não explica onde ou como elas serão armazenadas. No que tange as informações de memória, como os 
endereços de memória são os do host, estes precisam ser traduzidos para que a análise forense seja feita. Segundo a comunidade, tal estratégia é imune a técnicas anti-
forenses empregadas por usuários maliciosos pois esta localizada fora da máquina virtual. Como a abordagem não tem conhecimento do que esta rodando dentro da máquina 
virtual precisade uma copia bit a bit da evidência. Embora pareça possível, não descreve como lida com o cenário onde uma máquina é despejada do pool e os recursos liberados. 

Quando comparado este trabalho, a presente proposta tem por vantagens ser um arcabouço para coleta e armazenamento de evidências. Esta abordagem usa uma estratégia 
diferente pois coleta-se a memória diretamente dentro da máquina virtual. Nessa estratégia evita-se o problema do gap semântico próprio das soluções por introspecção, 
não precisa realizar tradução de endereços de memória para viabilizar a análise forense mas, de acordo com a comunidade é mais sucetível a técnicas anti-forenses. 
a presente proposta tem também a vantagem de conhecer o que esta rodando dentro da máquina e assim ter mais eficiência na coleta.\\

\item \textbf{Design and implementation of FROST: FoRensic tools for Open STack \cite{Dykstra2013} }: Framework para coleta de dados de máquinas virtuais através da API do
hypervisor. Isola a máquina virtual afetada do pool original para realização da coleta. Precisa ser acionado quando uma ameaça é detectada. É o mais bem acabado arcabouço de
todas as propostas encontradas até agora mas ao detalhar o processo de armazenamento não explica como garante que a evidência não será destruida ou alterado no transporte
até o armazenamento nem como controla o acesso a evidência. Por estar integrado ao Open Stack o arcabouço depende de cooperação do provedor de serviços de núvem onde ele 
esta rodando, isso é considerado problemático pela comunidade pois a prioridade do mesmo é manter o serviço funcionando e não coletar evidencias forenses. Como esta na
mesma camada do hypervisor não conhece o que esta rodando dentro da máquina. Depende da existência da máquina virtual para realização da coleta.

Quando comparado a este trabalho, a presente proposta tem por vantagens a utilização de container para associar a evidência de memória a sua origem tornando o processo 
independente de máquina e a janela de x dias antes da detecção do ataque para conseguir descrever o sistema antes do mesmo. Esta abordagem não depende de cooperação do 
provedor do serviço de nuvem. Tem também a vantagem de conhecer o que esta rodando dentro da máquina e assim conseguir mais eficiência na coleta.\\
  
\item \textbf{Automated Forensic Data Acquisition in the Cloud \cite{Reichert2015} }: Propõe um modelo que tira snapshots de máquinas virtuais atrelado a algum mecanismo de
detecção de ameaça baseado no hypervisor. Usa o Google Rapid Response para salvar as informações coletadas fora da núvem de forma a driblar os problemas de multi-jurisdição e 
multi-inquilino. Descreve satisfatóriamente a cadeia de custódia da evidência.

O modelo proposto só começa a coletar evidência após a detecção da ameaça e toma um snapshot da máquina toda o que já foi julgado pela comunidade como um processo custoso em
termos de espaco em disco pois piora o problema do volume de dados a ser analisado. Pessoalmente acho ariscado depender de snapshots pois caso precise, repetir o processo de coleta
pode não ser possível. Um exemplo é editar um HD virtual que esta atrelado a uma máquina virtual da qual se gerou snapshots, tal ação pode levar a perda de dados.

Quando comparado a este trabalho, a presente proposta tem por vantagens coletar apenas a informações de memória e usar a janela de coleta de x dias antes do ataque para 
manter sob controle a quantidade de informação que precisa ser analisada. Propondo a autilização de container para associar a evidência a sua origem, torna o processo 
independente de máquina. \\
 
\item \textbf{A log based approach to make digital forensics easier on cloud computing \cite{Sang2013} }: Método que sugere salvar a informação coletada fora da núvem de modo 
a driblar os problemas de multi-inquilinato e multi-jurisdição, usa um mecanismo de hash para garantir a autenticidade e integridade da informação mas não dá detalhes da 
implementação e não descreve como controla o acesso a evidência armazenada. Segundo o próprio autor, o método não funciona em IaaS. Precisa da cooperação do provedor de 
nuvem pois depende das informações que este último decidiu adicionar ao log. O método não é aplicável a coleta de informações de memória.

A proposta não coleta dados de memória por decisão do autor, esta proposta entrou na lista pela abordagem baseada em log. Neste quesito, a presente proposta é melhor quando 
garante todos os pontos da cadeia de custódia, a evidência não foi alterada ou destruida no transporte e o acesso a mesma é controlado. No âmbito da informação coletada 
a atual proposta não depende das decisões do provedor de nuvem sobre o que guardar no log para conseguir a evidência. \\

\item \textbf{Volatile memory acquisition using backup for forensic investigation \cite{Dezfouli2012} }: Técnica desenvolvida para dispositivos móveis que sugere a utilização
do próprio como repositório das evidências coletadas da memória. Para manter a utilização de espaço ao mínimo sugere manter apenas o último estado conhecido da memória.
 
É uma técnica interessante do ponto de vista de estratégia de armazenamento quando guarda apenas o último estado da memória. Essa abordagem porém perde a informação do momento 
do ataque e não consegue descrever o sistema antes do mesmo. Do resto da proposta não é aplicável para este projeto pois, armazenando a evidência na máquina a mesma seria perdida
quando a máquina fosse despejada do pool e seus recursos liberados. A cadeia de custódia não é abordada na proposta.\\

\item \textbf{Narrowing the semantic gap in virtual machine introspection \cite{Dolan-Gavitt2011a} }: Esta proposta é uma combinação da técnica de introspecção em máquina virtual
e a integração com a API do hypervisor. O principal objetivo é diminuir o gap semântico para facilitar a análise da evidência. Para isso o autor implementa um API para transformar
dados de baixo nível em informação de alto nível. Depende de cooperação do provedor de serviço de nuvem, não tem conhecimento da máquina hospedada e não vai além da coleta, não 
descreve como resolve a cadeia de custódia. Tem a vantagem de ser imune a técnicas anti-forenses.

Quando comparado a este trabalho, a presente proposta tem por vantagens ser arcabouço de coleta e armazenamento de evidências não apenas um método de coleta. Empregando a
estratégia de realizar a coleta diretamente na máquina não tem o problema do gap semântico próprio das soluções baseadas em introspecção e como conhece o contexto do que 
esta rodando dentro da máquina virtual consegue ser mais eficiênte na coleta.\\

\item \textbf{Digital Forensics as a Service: A game changer \cite{VanBaar2014} }: Esta proposta é focada em uma mudança no armazenamento e forma de trabalho dos peritos forenses.
Propoe que a forense seja oferecida como um serviço e que as evidências sejam armazenadas em um local centralizado com o devido controle de acesso e garantia de integridade da 
evidência. Descreve a arquitetura de armazenamento, qual o perfil que deve ter acesso a evidência e como é este acesso. Esta proposta não é focada apenas em incidentes na nuvem
mas em qualquer outro incidente.

Embora seja uma ótima proposta de armazenamento de evidências e controle de acesso a elas, ele não descreve o processo de coleta nem de transporte. É uma proposta focada mais na 
solução de problemas relacionados a manipulação da informação após a coleta, transporte e armazenamento. Não toca no assunto de coleta qualquer que seja, em nuvem ou fisica.\\

\item \textbf{Live Digital Forensics in a Virtual Machine \cite{Zhang2010} }: Proposta para coletar informações de memória de máquinas virtuais através de snapshots das mesmas.
O metodo de coleta envolve tomar o snapshot da máquina, no diretório onde o snapshot foi armazenado pegar o arquivo referente a memória e abri-lo / analisá-lo usando um programa de
leitura de memória de mercado. O autor não trás detalhes do transporte, armazenamento ou cadeia de custódia. Precisa que a máquina exista para conseguir coletar a informação e 
o processo é dependente de intervenção humana. Analisando com mais cuidado é possível repetir a coleta mesmo sem a máquina existir uma vez que temos o snapshot mas o autor não
dá detalhes do caso.

Quando comparado a este trabalho, a presente proposta tem por vantagens a menor quantidade de informação necessária à investigação através da implementação da janela de x
dias antes do incidente. Como o processo é automático, uma vez disparado não requer intervenção humana. A presente proposta descreve como garante a cadeia de custódia da 
evidência e consigue reproduzir o processo de coleta mesmo se a máquina não existir mais pois a evidência esta atrelada ao container.


 \end{itemize}
 
% ----------------------------------------------------------
% Análise dos resultados
% ----------------------------------------------------------
\chapter{Análise dos Resultados}

Aqui a análise dos resultados

% ---
% Finaliza a parte no bookmark do PDF
% para que se inicie o bookmark na raiz
% e adiciona espaço de parte no Sumário
% ---
\phantompart

% ----------------------------------------------------------
% ELEMENTOS PÓS-TEXTUAIS
% ----------------------------------------------------------
\postextual

% ----------------------------------------------------------
% Referências bibliográficas
% ----------------------------------------------------------
\bibliography{abntex2-modelo-references}

%---------------------------------------------------------------------
% INDICE REMISSIVO
%---------------------------------------------------------------------

\phantompart

\printindex


\end{document}
